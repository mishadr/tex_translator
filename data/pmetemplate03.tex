%%%%%%%%%%%%%%%%%%%%%%%%%%%%%%%%%%%%%%%%%%%%%%%%%%%%%%%
% % Lines starting with % are comments, which are ignored.
% % This is a handy way of indicating the date and version of
% % your document, to wit:
% %
% % LaTeX sample file
% % Modified March, 2002
% %
%%%%%%%%%%%%%%%%%%%%%%%%%%%%%%%%%%%%%%%%%%%%%%%%%%%%%%%
% % Title and author(s)
%%%%%%%%%%%%%%%%%%%%%%%%%%%%%%%%%%%%%%%%%%%%%%%%%%%%%%%
\title{The \LaTeX\  template file for an article in the
       $\Pi$ME Journal}
\author{Primus Scriber\thanks{
                  College of the Enlightenment}
        \and
        Theeco Author\thanks{
                      The Virtual University
                  }\thanks{This work was
                  supported by NSB grant number G983578765401.}
        }
%%%%%%%%%%%%%%%%%%%%%%%%%%%%%%%%%%%%%%%%%%%%%%%%%%%%%%%
\documentclass{article}
%%%%%%%%%%%%%%%%%%%%%%%%%%%%%%%%%%%%%%%%%%%%%%%%%%%%%%%
% %
% % The next command allows your in import encapsulated
% % postscript files, .epsf or .eps files, which
% % contain vector graphic image data.
% %
%%%%%%%%%%%%%%%%%%%%%%%%%%%%%%%%%%%%%%%%%%%%%%%%%%%%%%%
\usepackage{graphicx}
%%%%%%%%%%%%%%%%%%%%%%%%%%%%%%%%%%%%%%%%%%%%%%%%%%%%%%%
% % We use newtheorem to define theorem-like structures
% %
% % Here are some common ones. . .
%%%%%%%%%%%%%%%%%%%%%%%%%%%%%%%%%%%%%%%%%%%%%%%%%%%%%%%
\newtheorem{theorem}{Theorem}
\newtheorem{lemma}{Lemma}
\newtheorem{proposition}{Proposition}
\newtheorem{scolium}{Scolium}   %% And a not so common one.
\newtheorem{definition}{Definition}
\newenvironment{proof}{{\sc Proof:}}{~\hfill QED}
\newenvironment{AMS}{}{}
\newenvironment{keywords}{}{}
%%%%%%%%%%%%%%%%%%%%%%%%%%%%%%%%%%%%%%%%%%%%%%%%%%%%%%%
% %   The first thanks indicates your affiliation
% %
% %  Just the name here.
% %
% % Your mailing address goes at the end.
% %
% % \thanks is also how you indicate grant support
% %
%%%%%%%%%%%%%%%%%%%%%%%%%%%%%%%%%%%%%%%%%%%%%%%%%%%%%%%


\begin{document}
\newpage
\maketitle
%%%%%%%%%%%%%%%%%%%%%%%%%%%
% abstract, keywords and Subject classification are optional.
%%%%%%%%%%%%%%%%%%%%%%%%%%%
\begin{abstract}
    This is a sample file.
    You can use it as a guide for your submission.
\end{abstract}

% Most people don't use these, so they are "commented out"
% by starting the lines with a "%"
%\begin{keywords}
%   \LaTeX, typesetting
%\end{keywords}

%\begin{AMS}
%   50C60, 18C25
%\end{AMS}

%%%%%%%%%%%%%%%%%%%%%%
% % Here is the start of the Text
%%%%%%%%%%%%%%%%%%%%%%
\section{\LaTeX}
Use sectioning commands for headings. Often longer articles are divided into a few sections.


\LaTeX knows that a new paragraph has started if you skip a line in the input file.
It will automatically indent the proper amount.

You must use labelling commands, e.g. \verb2\label2, \verb2\ref2,
\verb2\bibitem2, and \verb2\cite2, to refer to sections of your document, such as
see Section~\ref{AbelSection}, see Figure~\ref{SelfdualFig}, or bibliography entries,
such as see~\cite{MyFavorite}.  Otherwise the look of the numbers, and sometimes the numbers
themselves, will be wrong in the final version at the printer.

\section{Abel's Theorem}\label{AbelSection}
     In this section we give some background.  For instance
the following definition.

\begin{definition}
     A {\em semi-quaver} is defined to be half a quaver.
\end{definition}

If that definition is not enough, here is another:

\begin{definition}
    The {\em order} of a note $n$ in a quaver $Q$,
    $O(n,Q)$, is defined by the equation
         $$ O(n,Q) = \int_{0}^{\infty} \sin(n^{2} t)/(1-Qn) dt. $$
\end{definition}

Some equations one writes inline, such as
{\em Pythagoras' Theorem,}
$c^{2} = a^{2} + b^{2}$,
while others are better off as displayed
equations, like the quadratic formula,
  $$x = \frac{-b \pm \sqrt{b^{2} - 4ac}}{2a}$$
which solves the quadratic $ax^{2} + bx + c = 0$. If the quadratic
formula is written inline,
  $x = \frac{-b \pm \sqrt{b^{2} - 4ac}}{2a}$,
it is readable but not very nice.
This form,
  $x = (-b \pm \sqrt{b^{2} - 4ac})/2a$,
is harder to read as a fraction, but better because of the larger type.

Every inline equation must be part of a sentence:  Since $x < 1/2$ we have
$x + y < x + 1/2$.  Inline fractions, such as $x < \frac{1}{2}$, are discouraged but
not prohibited.


You can use formulae in theorems, as in the following.

\begin{theorem}\label{ContThrm}
    If $f(x)$ is defined by the equation
    \begin{equation}\label{SplitFunc}
         f(x) = \left\{ \begin{array}{rl}
                          x^{2}, & \mbox{~for $x \geq 0$.} \\
                         -x^{2}, & \mbox{~for $x < 0$}
                         \end{array} \right.
   \end{equation}
   Then $f(x)$ is continuous at $x=0$.
\end{theorem}


\begin{proof}
    Since
          $\lim_{x \rightarrow 0^{-}} f(x) =
           \lim_{x \rightarrow 0^{-}} -x^{2} = 0$
    and
          $\lim_{x \rightarrow 0^{+}} f(x) =
           \lim_{x \rightarrow 0^{+}} x^{2} = 0$
    it follows that
         $\lim_{x \rightarrow 0} f(x) = 0 = f(x)$,
         as required.
\end{proof}





   Did you notice the grammatical error in
   Theorem~\ref{ContThrm}?  The sentence leading
   into equation~\ref{SplitFunc} is never completed.
   %%%% Notice references are done with \ref
   %%%% you can label any structure that produces a number
   The following
   theorem is worded correctly.
\begin{theorem}
    If $f(x)$ is defined by the equation
    \begin{equation}
         f(x) = \left\{ \begin{array}{rl}
                          x^{2}, & \mbox{~for $x \geq 0$} \\
                         -x^{2}, & \mbox{~for $x < 0$}
                         \end{array}\right.,
   \end{equation}
   then $f(x)$ is continuous at $x=0$.
\end{theorem}

   One rule of thumb of mathematical composition is to
   use mathematical notation inside sentences only for
   nouns.  For example, one writes that ``$R$ is the the
   radius of a circle'', but not that ``the radius of the
   circle ='s the side length of the square''.  According to
   this rule it is correct to write that
   ``$(x>0) \Rightarrow (x^{3} > 0)$'' since the double arrow is part of
   an equation, but not to write
   ``$x$ is positive $\Rightarrow$ $x^{3}$ is positive'', since the
   double arrow is acting as a verb.

Here is another kind of common structure:

\begin{theorem}
   The following are equivalent.
   \begin{enumerate}
      \item $a \leq b$
      \item $b \geq a$
      \item $a = b$ or $a<b$
   \end{enumerate}
\end{theorem}

And here is a typical matrix
 $$
 \left[
 \begin{array}{rrrrrrrr}
    0 & -1 & 0 & 1 & 0 & 0 & 0 & 0  \\
    1 & 0 & 0 & 0 & -1 & 0 & 0 & 0  \\
    0 & 0 & \alpha & \alpha & -\alpha & -\alpha & 0 & 0  \\
    \alpha & \alpha & 0 & 0 & 0 & 0 & -\alpha & -\alpha  \\
    0 & 0 & 1 & 0 & 0 & 0 & -1 & 0  \\
    0 & 0 & 0 & 0 & 0 & 1 & 0 & -1
 \end{array}
 \right]
 $$


Sometimes you want to use a figure.
\begin{figure}[hbt]
     \centering
     \unitlength=0.75mm
     \begin{picture}(100.00,95.00)
          \put(20.00,50.00){\line(1,2){5.00}}
          \put(35.00,60.00){\line(1,-2){5.00}}
          \put(40.00,50.00){\line(-2,-1){10.00}}
          \put(30.00,45.00){\line(-2,1){10.00}}
          \put(20.00,50.00){\line(-1,0){15.00}}
          \put(40.00,50.00){\line(1,0){20.00}}
          \put(60.00,50.00){\line(2,1){10.00}}
          \put(70.00,55.00){\line(2,-1){10.00}}
          \put(80.00,50.00){\line(-1,-2){5.00}}
          \put(75.00,40.00){\line(-1,0){10.00}}
          \put(65.00,40.00){\line(-1,2){5.00}}
          \put(80.00,50.00){\line(1,0){15.00}}
          \put(50.00,40.00){\line(0,1){20.00}}
          \put(50.00,40.00){\line(1,-2){5.00}}
          \put(55.00,30.00){\line(-1,-2){5.00}}
          \put(50.00,20.00){\line(-2,1){10.00}}
          \put(40.00,25.00){\line(0,1){10.00}}
          \put(40.00,35.00){\line(2,1){10.00}}
          \put(50.00,20.00){\line(0,-1){15.00}}
          \put(50.00,60.00){\line(-1,2){5.00}}
          \put(45.00,70.00){\line(1,2){5.00}}
          \put(50.00,80.00){\line(2,-1){10.00}}
          \put(60.00,65.00){\line(-2,-1){10.00}}
          \put(35.00,60.00){\line(3,2){15.00}}
          \put(60.00,65.00){\line(2,-3){10.00}}
          \put(65.00,40.00){\line(-3,-2){15.00}}
          \put(40.00,35.00){\line(-2,3){10.00}}
          \put(45.00,75.00){\line(1,-1){5.00}}
          \put(75.00,55.00){\line(-1,-1){5.00}}
          \put(75.00,40.00){\line(-1,-1){15.00}}
          \put(60.00,25.00){\line(-1,0){5.00}}
          \put(55.00,25.00){\line(-1,1){5.00}}
          \put(40.00,25.00){\line(-1,1){15.00}}
          \put(25.00,40.00){\line(0,1){5.00}}
          \put(25.00,45.00){\line(1,1){5.00}}
          \put(25.00,60.00){\line(1,0){10.00}}
          \put(60.00,65.00){\line(0,1){10.00}}
          \put(60.00,75.00){\line(1,-1){15.00}}
          \put(75.00,60.00){\line(0,-1){5.00}}
          \put(25.00,60.00){\line(1,1){15.00}}
          \put(40.00,75.00){\line(1,0){5.00}}
          \put(30.00,50.00){\line(2,1){20.00}}
          \put(50.00,70.00){\line(1,-2){10.00}}
          \put(70.00,50.00){\line(-2,-1){20.00}}
          \put(50.00,30.00){\line(-1,2){10.00}}
          \put(50.00,70.00){\line(1,1){10.00}}
          \put(60.00,80.00){\line(1,0){10.00}}
          \put(70.00,80.00){\line(1,-1){10.00}}
          \put(80.00,70.00){\line(0,-1){20.00}}
          \put(70.00,50.00){\line(1,-1){10.00}}
          \put(80.00,40.00){\line(0,-1){10.00}}
          \put(80.00,30.00){\line(-3,-2){15.00}}
          \put(65.00,20.00){\line(-1,0){15.00}}
          \put(50.00,30.00){\line(-1,-1){10.00}}
          \put(40.00,20.00){\line(-1,0){10.00}}
          \put(30.00,20.00){\line(-1,1){10.00}}
          \put(20.00,30.00){\line(0,1){20.00}}
          \put(30.00,50.00){\line(-1,1){10.00}}
          \put(20.00,60.00){\line(0,1){10.00}}
          \put(20.00,70.00){\line(1,1){10.00}}
          \put(30.00,80.00){\line(1,0){20.00}}
          \put(50.00,70.00){\line(3,-1){15.00}}
          \put(65.00,65.00){\line(1,-2){5.00}}
          \put(70.00,50.00){\line(-1,-3){5.00}}
          \put(65.00,35.00){\line(-2,-1){10.00}}
          \put(50.00,30.00){\line(-3,1){15.00}}
          \put(35.00,35.00){\line(-1,2){5.00}}
          \put(30.00,50.00){\line(1,3){5.00}}
          \put(35.00,65.00){\line(2,1){10.00}}
          \put(20.00,50.00){\circle*{3.00}}
          \put(30.00,45.00){\circle*{3.00}}
          \put(40.00,50.00){\circle*{3.00}}
          \put(35.00,60.00){\circle*{3.00}}
          \put(25.00,60.00){\circle*{3.00}}
          \put(50.00,70.00){\circle*{3.00}}
          \put(70.00,55.00){\circle*{3.00}}
          \put(80.00,50.00){\circle*{3.00}}
          \put(75.00,40.00){\circle*{3.00}}
          \put(65.00,40.00){\circle*{3.00}}
          \put(60.00,50.00){\circle*{3.00}}
          \put(50.00,30.00){\circle*{3.00}}
          \put(45.00,70.00){\circle{3.00}}
          \put(50.00,80.00){\circle{3.00}}
          \put(60.00,75.00){\circle{3.00}}
          \put(60.00,65.00){\circle{3.00}}
          \put(30.00,50.00){\circle{3.00}}
          \put(40.00,35.00){\circle{3.00}}
          \put(40.00,25.00){\circle{3.00}}
          \put(50.00,20.00){\circle{3.00}}
          \put(55.00,30.00){\circle{3.00}}
          \put(50.00,40.00){\circle{3.00}}
          \put(70.00,50.00){\circle{3.00}}
          \put(50.00,60.00){\circle{3.00}}
          \put(50.00,80.00){\line(0,1){10.00}}
          \put(50.00,95.00){\makebox(0,0)[cc]{$\vdots$}}
          \put(50.00,0.00){\makebox(0,0)[cc]{$\vdots$}}
          \put(100.00,50.00){\makebox(0,0)[cc]{$\ldots$}}
          \put(0.00,50.00){\makebox(0,0)[cc]{$\ldots$}}
     \end{picture}
     \caption{A polyhedron with all self--dualities of
              order~4\label{SelfdualFig}}
\end{figure}
Figure~\ref{SelfdualFig} was made (using TeXCad)
with the \LaTeX picture
environment.  Sometimes a figure is just too
complicated to draw with these simple commands.  In this case
we use epsf files.
For instance in Figure~\ref{ParabFig} the picture of the
parabola was produced in Maple
and saved as an epsf (eps) file, (pure ascii mode, no
preview, no thumbnail).
  \begin{figure}[htb]
      \centering
      \includegraphics[width=2.5in]{para01.eps}
      \caption{$y = x^{2}$\label{ParabFig}}
  \end{figure}
When the document is processed, the .eps file must reside
in the same directory as the .tex file.  Also, the command
\verb2\usepackage{graphicx}2 should occur near the top of the document.
Notice that this figure has been scaled so that the overall size
is convenient (width 2.5 inches), but now the text is far too small.
To avoid this problem it is often best to
generate a figure which is approximately the same size as
it will appear in the document.

A complete source of information on writing documents
in \LaTeX is~\cite{MyFavorite}.  (Look in the source
so see how to produce that citation.)
Last of all is the style of the $\Pi$ME biliography:
Author's names in small caps, journal article titles
uncapitalized and in italics, book titles capitalized
and in quotes.

\begin{thebibliography}{9}
     \bibitem{MyFavorite}
         {\sc Lamport, L.,}
         ``\LaTeX - A Document Preparation System'',
         Addison-Wesley, 1998.

    \bibitem{BobsPaper}
         {\sc Fillioque R.} and {\sc Heliotrope, B.,}
         {\em Why Fermat's last theorem is really a lemma,}
         American Mathematical Weekly,
         Vol. 7, No. 1, pp 115-116, 1998.

\end{thebibliography}



\section*{About the author:}
   We would like a short biographical sketch,
   beyond just your affiliation to be placed
   after the bibliography.
   And below that, your full address.



\subsection*{Primus Scriber}
   College of the Enlightenment,
   Philadelphia, Pennsylvania, 42345-6543$\pm\epsilon$.
   pscriber@cenet.edu

\subsection*{Theco Author}~
   Department of Statistics,
   The Virtual University,
   New York, NY 13291-5555.
   also@aol.com

\end{document}
